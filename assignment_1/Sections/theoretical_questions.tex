\section{Theoretical Questions}

\begin{enumerate}
  \item 
    AI is a broad field with many interpretations, and there are thus 
    numerous ways to define this field. One broad definition is that AI is the 
    study and construction of agents that do the right thing. Though what is 
    considered the right thing is also up for debate. An AI should be able to 
    make decisions or act in a way that could mimic human intelligence, thus it 
    should emit intelligent behavior. Moreover, what we may consider intelligent
    behavior is possible to define in many ways. One way to define it is that an
    AI should try to maximize expected utility. This means that the AI should 
    try to achieve some goal within a given set of contraints in a way that 
    maximizes the expected outcome \autocite[pp.~7--22]{russell2021artificial}.
  \item 
    The Turing test is an experiment created by Alan Turing to see if a 
    machine could exhibit intelligent enough behaviour to such an extent that
    a human interacting with it, for instance posing and answering questions,
    could not distinguish it from another human. A new proposal of the Turing
    test includes interaction not only through answering and posing questions,
    but actual physical interaction with humans and objects in the real world,
    which would require for instance computer vision and robotics in addition
    to natural language processing \autocite[p.~23]{russell2021artificial}.
  \item 
    Rationality is defined as doing the right thing. Thinking rationally can be 
    described as the ability to make predictions about the future based on a 
    sound understanding of how the world works. Acting rationally, however, is 
    the ability to make decisions that maximize expected utility. An AI doesn't 
    need to be able to perform rational thinking the way we humans do to act 
    rationally. It should make decisions that are optimal given a set of goals 
    or constraints, and may act rationally without rational though if it for 
    example has learned from past runs or follows a set of rules 
    \autocite[pp.~19--22]{russell2021artificial}.
  \item
    According to Aristotle, there is a connection between knowledge and action
    due to the fact that to reach a goal, one has to have knowledge about the 
    outcomes of an action, and thus what actions are required to reach a 
    specific goal. Aristotle himself proposed an algorithm which explains how
    his argument can be used to implement the idea in AI. One would assume the 
    goal state and then work backwards to find a sequence of actions that would
    lead to that goal state, i.e. regression.
    \begin{enumerate}
        \item 
            Newell and Simon were the first AI researchers to implement 
            Aristotle's ideas.
        \item 
            The name of the program they developed was General Problem Solver, 
            which is a planning system that uses a greedy algorithm together
            with regression to find a solution to a problem.
    \end{enumerate} 
    \autocite[p.~25]{russell2021artificial}
  \item 
    \begin{itemize}
    \item
        The robot has a set of restricted actions, which include looking and 
        going in four different directions. We can assume that the robot would 
        look in a specific direction, e.g. forward, to see if there were any 
        obstacles blocking it from reaching its goal, i.e. crossing the road. 
        If there were, it would take a step left/right, and if there were none, 
        simply go forward. It is after this action is taken that an elk 
        suddently crashes into the robot and smashes it. Based on this, the 
        robot made decisions based on its available actions and the information 
        it had at the time, and thus acted rationally. The robot could not have 
        known that an elk would crash into it, and thus could not have avoided 
        it. 
    \item
        The robot chose to cross the road on a green light, which from using its 
        lookForward action, would mean that it was safe to cross the road. 
        Again, the robot could not have known that a car would drive a red light 
        and subsequently crash into it, and thus acted rationally. Even though 
        it didn't reach its goal in any cases, it still acted within the actions 
        and knowledge it had at its disposal to best reach its goal.
    \end{itemize}
  \item 6
  \item 7
  \item 8
\end{enumerate}

